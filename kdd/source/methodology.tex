\section{Methodology}
\label{sec:methodology}

For this work we have applied a multitude of different approaches for detecting anomalies/outliers.
In this section, we detail the specific algorithms employed from these various categories.

\subsection{Cluster Cardinality}

For this method we equate anomalousness with the cardinality of the cluster to which a point belongs.
If a point belongs to a cluster with a very small number of points, it is more likely that the point is an anomaly or an outlier.
Clusters that have a large cardinality are treated to be more likely to be normal.

\begin{algorithm}
\DontPrintSemicolon
\SetAlgoLined
\KwData{manifold, depth}
\KwResult{scores}
 $scores \leftarrow \{\}$\;
 $d \leftarrow depth$\;
 \ForAll{cluster $c \in$ manifold.graph(d).clusters}{
  \ForAll{$p \in c.points$}{
   $scores[p] \leftarrow |c|$
  }
 }
 normalize(scores)\;
 \ForAll{$p \in scores$}{$scores[p] \leftarrow 1-p$}
 \caption{Cluster Cardinality}
 \label{alg-cc}
\end{algorithm}

\subsection{Hierarchical}

For this method, examine the ratio between the cardinality of a cluster and of it's parent.
A point is considered to be more anomalous if it's parent cluster is large and it's cluster is small.
Intuitively, this relationship would indicate that the point potentially sits off of the manifold.

\begin{algorithm}
\DontPrintSemicolon
\SetAlgoLined
\KwData{manifold,depth}
\KwResult{scores}
 $scores \leftarrow \{\}$\;
 $d \leftarrow depth$\;
 \For{$d \in 1\cdots$ depth}{
  \ForAll{cluster $c \in $manifold.graph(d).clusters}{
    \ForAll{point $p \in c$}{
     $scores[p] \leftarrow \frac{|c.parent|}{|c|\times d}$
    }
  }
 }
 normalize(scores)\;
 \caption{Hierarchical}
 \label{alg-hierarchical}
\end{algorithm}

\subsection{k-Neighborhood}

For this approach, we examine the number of clusters reachable within $k$-steps of a cluster.
The more clusters which are reachable from a given cluster, the more likely it is to not contain anomalous points.

\begin{algorithm}
\DontPrintSemicolon
\SetAlgoLined
\KwData{manifold,depth,k}
\KwResult{scores}
 $scores \leftarrow \{\}$\;
 $d \leftarrow $ depth\;
 $g \leftarrow $ manifold.graph(d).clusters\;
 \ForAll{cluster $c \in g$}{
  $v \leftarrow |\{c' \in g | \delta(c,c') \le k\}|$\;
  \ForAll{$p \in c$}{
   scores[p] $\leftarrow v$\;
  }
 }
 normalize(scores)\;
 \ForAll{$p \in scores$}{$scores[p] \leftarrow 1-p$}
 \caption{k-Neighborhood}
 \label{alg-kneighborhood}
\end{algorithm}

\subsection{Random Walk}

The general idea behind this approach, which is derived from Outrank~\cite{moonesinghe_outrank:_2008}, is to examine the frequency with which clusters are visited during $n$ random walks.
Clusters that are visited more frequently, are again less likely to be anomalous.
Clusters that are not visited frequently are more likely to be disconnected, and distant from the underlying manifold.

\begin{algorithm}
\DontPrintSemicolon
\SetAlgoLined
\KwData{manifold,depth}
\KwResult{scores}
 $scores \leftarrow \{\}$\;
 $d \leftarrow $ depth\;
 $g \leftarrow $ manifold.graph(d).components\;
 \ForAll{component $k \in g$}{
  visits $\leftarrow$ outrank(k)\;
  \ForAll{cluster $c \in k$}{
   $v \leftarrow $visits[c]\;
   \ForAll{$p \in c$}{
    scores[p] $\leftarrow v$\;
   }
  }
 }
 normalize(scores)\;
 \ForAll{$p \in scores$}{$scores[p] \leftarrow 1-p$}
 \caption{Random Walk}
 \label{alg-rw}
\end{algorithm}

\subsection{Subgraph Cardinality}

In this approach we utilize the cardinality of each connected subcomponent of the graph at various depths to determine anomalousness.
Similarly to some of our other approaches, we postulate that clusters which are members of sparse components are themselves more likely to be anomalous.

\begin{algorithm}
\DontPrintSemicolon
\SetAlgoLined
\KwData{manifold, depth}
\KwResult{scores}
 $scores \leftarrow \{\}$\;
 $d \leftarrow depth$\;
 \ForAll{component $k \in$ manifold.graph(d).components}{
  \ForAll{cluster $c \in k$}{
   \ForAll{$p \in c$}{
    scores[p] $\leftarrow |k|$
   }
  }
 }
 normalize(scores)\;
 \ForAll{$p \in scores$}{$scores[p] \leftarrow 1-p$}
 \caption{Subgraph Cardinality}
 \label{alg-sgc}
\end{algorithm}

\subsection{Normalization Methods}

Due to the wide range of possible measurements for ``anomalousness'' from our methods, we normalize our measurements.

\begin{enumerate}
    \item Min-Max Scaling:
    \begin{gather}
        x^{\prime} = \frac{x - x_{min}}{x_{max} - x_{min}}
        \label{sec:methods:min-max-normalizationn}
    \end{gather}
    
    \item Mean-Scaling:
    \begin{gather}
        x^{\prime} = \frac{x - \overline{x}}{x_{max} - x_{min}}
        \label{sec:methods:mean-scaling}
    \end{gather}
    
    \item z-Score Standardization:
    \begin{gather}
        x^{\prime} = \frac{x - \overline{x}}{\sigma}
        \label{sec:methods:z-score}
    \end{gather}

\end{enumerate}
