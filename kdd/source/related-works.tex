\section{Related Works}
\label{sec:related_works}

In this section, we consider previous works in distance-based, clustering-based, and graph-based approaches in outlier/anomaly detection.

\subsection{Clustering-Based Approaches}

Clustering data is universally centered around techniques for grouping instances in a way that provides value.
This is generally done by assigning \textit{similar} points to the same cluster.
Once clusters are formed, they can be used to inspect the data in great detail.
For example, for any given point, one could estimate it's anomalousness by it's distance to it's nearest cluster.
To determine if a cluster contains outliers, one could examine the relative cardinality of the cluster.

We summarize several major approaches to clustering data.
Note that there have been relatively few advancements in clustering techniques over the past decade~\cite{wang_progress_2019}. This may explain, or by attributable to, the observably poor performance, thus far, of clustering data in higher dimensional space~\cite{zhang_advancements_2013}.

\subsubsection{Distance-Based Clustering}

Distance-based clustering approaches rely on some distance measure to partition the data into some number of clusters.
Within this approach, the numbers of clusters and/or the sizes of clusters are normally predefined: either user-specified, or chosen at random~\cite{wang_progress_2019}.
Some examples of distance-based clustering are: 
K-Means~\cite{macqueen_methods_nodate}, 
PAM~\cite{kaufman_finding_nodate}, 
CLARANS~\cite{ng_efficient_nodate}, 
CLARA~\cite{kaufman_finding_nodate}, 
etc.

\subsubsection{Hierarchical Clustering}

Hierarchical clustering methods typically utilize a tree-like structure, where points are allocated into leaf nodes~\cite{wang_progress_2019}.
These tree-like structures can be created bottom-up (agglomerative clustering), or top-down (divisive clustering)~\cite{rakesh_agrawal_automatic_1998}.
The major drawback of these methods is the high cost of pairwise difference computations at each level of the tree.
Examples of hierarchical clustering include: 
MST~\cite{charles_zahn_graph_1971}, 
CURE~\cite{noauthor_cure:_nodate}, 
CHAMELEON~\cite{karypis_chameleon:_nodate}, 
etc.

\subsubsection{Density-Based Clustering}

Density-Based clustering methods rely on finding areas of high point-density separated by areas of low point-density.
These algorithms generally do not work well when normal data are sparse or uniformly distributed.
Some examples of density-based clustering algorithms are: 
DBSCAN~\cite{ester_density-based_nodate}, 
DENCLUE~\cite{Hinneburg1998Effic-5816}, 
etc.

\subsubsection{Grid-Based Clustering}

Grid-based clustering works via segmenting the entire space into a finite number of cells and then iterating over the cells to find regions of higher density.
Utilizing the grid structure for clustering means that these algorithms have typically scaled better to larger datasets.
Some examples of grid-based clustering: 
STING~\cite{sting:wang}, 
Wavecluster~\cite{Wavecluster:Sheikholeslami:2000}, 
DCluster,  % Paper does not exist????
CLIQUE~\cite{rakesh_agrawal_automatic_1998}, 
etc.

\subsection{Distanced-Based Approaches}

Distance-based methods attempt to find anomalous points via distance comparisons.
These methods largely employ k-Nearest Neighbors as their substrate~\cite{wang_progress_2019}.
They proceed in the following, slightly different, ways:

\begin{enumerate}
    \item Points with less than $p$ other points within a distance $d$ are outliers.
    \item The top $n$ points whose distances to their $k^{th}$-nearest neighbor are the greatest are outliers.
    \item The top $n$ points whose average distances to their $k^{th}$-nearest neighbor are the greatest are outliers.
\end{enumerate}


% CLIQUE Summary
\begin{comment}
CLIQUE scales linearly with input records, but does identify subspaces within the original dataset.
CLIQUE clusters based on subspaces it identifies via density estimates, it does not cluster in the full dimensionality of the input data.
This seems to be the premonition of manifold finding.
Restricted look to subspaces of the original set, not fictitious subspaces dreamed up via techniques such as Principle Component Analysis.
CLIQUE finds areas of higher density, which are the clusters.
Begins by partitioning the full space into units then estimates density within that cell.
All clusters are axis-parallel hyper-rectangles. Each cluster is a union of adjacent dense cells.
Compact representation is then found by covering the cluster with a minimal number of maximal, possibly overlapping rectangles and describing the cluster as the union of said rectangles.
CLIQUE is also tolerant to incomplete data points.
Requires density threshold and number of intervals over the full space as parameters.
Categorical data is ordered arbitrarily, with an empty interval placed in-between ever pair.
Is deterministic, always yields the same results.
Runs in $O(c^k + m k)$ where $c$ is some constant, $k$ is the dimensionality, and $m$ is the number of points.
Tested on dimensionality up to 50.
In experiments run for this paper, BIRCH failed starting at 20 dimensions, DBSCAN at 8, CLIQUE made it to 50 without failure.
\cite{rakesh_agrawal_automatic_1998}
\end{comment}
