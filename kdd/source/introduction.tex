\section{Introduction}
\label{sec:introduction}

Anomaly detection is difficult.
To determine if something deviates from normal, one must understand the normal, the abnormal, and the boundary between the two.

Normal data in the real-world are generated by some constrained generating phenomena.
Examples include:
\begin{itemize}
    \item biological evolution.
    \item stellar evolution.
    \item drug synthesis.
    \item network interactions.
\end{itemize}

Abnormal instances of data, which can be called \textit{outliers} or \textit{anomalies}, arise from such varied sources as:
\begin{itemize}
    \item errors in measurements and data collection.
    \item noise in data mimicking real outliers.
    \item blurry boundaries between abnormal and normal behaviour.
    \item normal behaviour evolving into abnormal behaviour.
    \item novel never-before-seen instances in data.
    \item adversarial attacks.
\end{itemize}

In this paper, we tackle this challenge using a novel technique we call Clustered Approximate Manifold Learning (CLAM).
% TODO: and compare our performance against oft-used state-of-the-art anomaly-detection techniques.
Derived from the work done for CHESS~\cite{ishaq2019entropy}, we begin by divisively clustering the data until every point is within its own cluster, as a singleton.
We then delineate \textit{layers} of clusters at each depth in the tree.
Each layer is comprised of all clusters that would have been leaf nodes if the tree building where to have been halted at the given depth.

We then build a graph for each layer in the tree by creating edges between clusters that have overlapping volumes.
This process effectively learns the manifold that the data lie on at various resolutions, given by the depth of the layer.
Once we have learned a manifold, one can ask about the cardinality of various clusters at different depths, how connected a given cluster is, or even how often a cluster is visited over many different random walks across the manifold.

We test our methods on 26 real-world datasets. 
Each dataset contains a different amount of anomalous data, for a different domain.

We consider several different definitions for outliers and anomalies.

\begin{itemize}
    \item distance based: examining several classical distance-based definitions of outliers, relying on the fact that CLAM uses distance to cluster the data
    \item density based: examining the cardinality of clusters, postulating that clusters with relatively low cardinality contain outliers
    \item graph based: examining several graph-based methods for anomaly detection with graphs constructed from the layers of clusters
\end{itemize}

All forms of clustering hitherto have suffered from one defect or another.
The most common deficiencies are: the effective treatment of high dimensionality, interpretability of results, and/or scalability to exponentially growing datasets ~\cite{rakesh_agrawal_automatic_1998}.
With CLAM, we have largely alleviated these problems while also decoupling search-time from the size of the data being searched.
By defining edges via cluster-overlap, we build a graph representation of the manifold that the data occupy at various resolutions.
We then use our knowledge of this manifold to guide the detection of anomalies, outliers and novelties.

% TODO: Add definitions: Anomaly, Outlier, Novelty, etc.
